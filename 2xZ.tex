\documentclass[a4paper]{article}
\usepackage[utf8]{inputenc}
\usepackage{amsmath}
\usepackage{amssymb}
\usepackage{amsfonts}
\usepackage{amsthm}
\usepackage{mathrsfs} 
\usepackage{xcolor}
\usepackage{hyperref}
\usepackage{xparse}
\usepackage{bm}
\usepackage{bbm}
\usepackage{tikz}
\usepackage{centernot}
\setlength{\parskip}{\baselineskip}
\DeclareMathOperator{\sign}{sign}
\DeclareMathOperator{\Div}{div}
\DeclareMathOperator{\curl}{curl}
\DeclareMathOperator{\lcm}{lcm}
\DeclareMathOperator{\hcf}{hcf}
\DeclareMathOperator{\rank}{rank}
\DeclareMathOperator{\im}{Im}
\DeclareMathOperator{\Null}{nullity}
\DeclareMathOperator{\Span}{Span}
\DeclareMathOperator{\Var}{Var}
\DeclareMathOperator{\Cov}{Cov}
\DeclareMathOperator{\Log}{Log}
\DeclareMathOperator{\lamdba}{\lambda}
\DeclareMathOperator{\Hom}{Hom}
\DeclareMathOperator{\tr}{tr}
\DeclareMathOperator{\Int}{int}
\DeclareMathOperator{\Res}{Res}
\DeclareMathOperator{\supp}{Supp}

\hypersetup{
    colorlinks=true,
    linkcolor=blue,
    filecolor=magenta,
    urlcolor=blue,
    citecolor=blue
}

\newtheorem{thm}{Theorem}
\newtheorem{bnd}{Bound}

\newtheorem{result}{Result}[section]
\newtheorem{lem}[result]{Lemma}
\newtheorem{rmk}[result]{Remark}
\newtheorem{cor}[result]{Corollary}
\newtheorem{red}[result]{Reduction}

\theoremstyle{definition}
\newtheorem*{defn}{Definition}
\newtheorem*{con}{Construction}
\newtheorem*{que}{Question}

\DeclareMathOperator{\Solve}{\bm{\mathsf{Solved}}}

\newcommand{\hide}[1]{}
\newcommand{\edit}[1]{}%\textcolor{red}{#1}}
\newcommand{\rough}[1]{}%\textbf{\textcolor{blue}{#1}}}
\definecolor{darkgreen}{RGB}{75,150,75}
\newcommand{\review}[1]{}%\textcolor{darkgreen}{#1}}
\newcommand{\dc}[1]{}%\textcolor{orange}{dc: #1}}
\newcommand{\zh}[1]{}%\textcolor{blue}{zh: #1}}
\newcommand{\hides}[1]{}%1}
\newcommand{\pub}[1]{}%\textcolor{purple}{#1}}

\newcommand{\al}{\alpha}
\newcommand{\be}{\beta}

\title{Universal Transversals}
\author{Zach Hunter}
\date{November 2020}

\begin{document}

\maketitle

\section{Notation}

We will be using a lot of the notation seen in (https://arxiv.org/pdf/1804.05439.pdf). 

We will use the convention that for an alphabet $\Sigma$, that $\Sigma^* = \bigcup_{n=0}^\infty \Sigma^n$ is the set of all finite words on $\Sigma$, $\Sigma^\omega = \{x:\Bbb{N}\to\Sigma\}$ is the set of infinite word on $\Sigma$ which are indexed by the naturals, and $S^\infty = S^*\cup S^\omega$. For $a \in \Sigma^*, b \in \Sigma^\infty$ will let $a\bigstar b$ denote the concatenation of $a$ and $b$. (although sometimes we will simply write $ab$ instead) Lastly, for $a \in S^*$, we set $a^1 = a$ and then recursively define $a^{i+1} = a^i\bigstar a$.

For a set $S$, an $S$\textit{-maze}, $M$, is a quadruple $(Q,\delta,q_1,q_T)$, consisting of a countable set $Q$, a function $\delta:Q\times S \to Q$, and elements $q_1,q_T \in Q$. We can think of a maze as resembling a deterministic automata on the alphabet $S$, where $q_0$ is the initial state, and $q_T$ is the target state. We say that $M$ is \textit{solved} by a word $a = a_1,\dots \in S^\infty$ iff considering the sequence defined by $r_1 = q_1, r_{i+1} = \delta(r_i,a_i)$, there exists $t\ge 1$ such that $r_t = q_T$. We say a set of $S$-mazes, $\mathcal{F}$, is solved by a word $a \in S^\infty$, if each maze $M\in \mathcal{F}$ is solved by $a$.

\begin{rmk}\label{fstrat} If there exists a countable partition $\mathcal{F} = \bigcup_{i=1}^\infty F_i$ such that for each $i\ge 1$ and each $a \in S^*$, there exists $b=b(a,i) \in S^*$ such that $a \bigstar b$ solves $F_i$, then there exists $A \in S^\infty$ such that $A$ solves $\mathcal{F}$.
\begin{proof}
    Let $a_0$ be the empty word, which has finite length. Inductively, given that $a_i$ is finite, we may define $a_{i+1} = a_i \bigstar b(a_i,i)$, which will again be finite as $a_i,b(a_i,i)$ each belong to $S^*$. (respectively due to inductive hypothesis and the assumption of the remark) It also is clear by induction that $a_n$ solves $\bigcup_{i=1}^n F_i$, as it contains $a_{n-1}$ as a prefix. Defining $A$ as the direct limit of $a_n$ it is clear to see that $A$ should have countable length and that $A$ solves $\bigcup_{i=1}^\infty F_i = \mathcal{F}$.
\end{proof}
\end{rmk}

We may represent the infinite grid graph as the Cayley graph of $\Bbb{Z}^2$ using the quadruple $(R,L,U,D):=((1,0),(-1,0),(0,1),(0,-1))$ as our generating set. (so geometrically the first coordinate is our $x$-axis and the second is our $y$-axis) We will let $G$ denote this Cayley graph, and set $S=\{R,L,U,D\}$. For a vertex $v \in V(G)$ we will let $x_v$ denote the $x$-coordinate of $v$ and $y_v$ denote the $y$-coordinate of $v$.

For a subgraph $H$ of $G$, and $u,v \in V(H)$, we let $M_{H,u,v} = (Q,\delta,u,v)$ be the $S$-maze where $Q = V(H)$, and for each $s \in S$ and $v \in Q$ we have \[\delta(v,s) = \begin{cases}v+s &\textrm{if $(v,v+s) \in E(H)$}\\v &\textrm{otherwise.}\end{cases}\]
For a connected subgraph $H$ we say that $a \in S^\infty$ solves $H$ starting at $u \in V(H)$, if $a$ solves $\{M_{H,u,v}:\forall v \in V(H)\}$. Similarly, $a$ solves a set of subgraphs $\mathcal{H}$ starting at $u$ if $a$ solves $\bigcup_{H \in \mathcal{H}}\{M_{H,u,v}:\forall v \in V(H)\}$. For brevity, we may sometimes write $M_{H,v}$ in place of $M_{H,(0,0),v}$.

Sometimes, for a subgraph $H$, a vertex $u \in V(H)$ and $a \in S^*$, we will use $\delta_H(u,a)$ to denote where we end the walk which starts at $v$ and follows the instructions of $a$. (basically if $|a| =n$ then $\delta_H(u,a) = r_{n+1}$, if we use the language from earlier)

\section{\texorpdfstring{Solving subgraphs of $\Bbb{Z}\times \{0,1\}$}{Solving subgraphs of Z x {0,1}}}

We shall prove that there exists $a \in S^\infty$ that solves $\mathcal{H}$ starting at $(0,0)$, where $\mathcal{H}$ is the set of all connected subgraphs $H$ of the induced subgraph $G' := G[ \Bbb{Z}\times \{0,1\}]$ such that $(0,0) \in V(H)$ and for $n \in \Bbb{Z}$, there exist $v_n \in V(H)$ such that $x_{v_n} = n$. The methods outlined should work in general, but are noticably messier when we account for $H$ such that there projection onto the $x$-axis is like a ray. As this is supposed to be a quick sketch to inform you, I've decided to omit the general case as it was taking to long and becoming quite illegible.

We will find the following definitions useful.
\begin{defn} For $n\le m\in \Bbb{Z}$ and $H \in\mathcal{H}$, let $H' = H_{n,m}$ be the maximal subgraph of $H$ such that $(x,y) \in V(H')\implies x \in [n,m]$.
\end{defn}
\begin{defn} For a graph $H$, let $d_H(u,v)$ be the minimum number of edges required to make a path in $H$ from $u$ to $v$. Here, we use the convention that $d_H(u,v)$ if $u,v$ are not connected in $H$, or either do not belong to $V(H)$.
\end{defn}

At a  high level, our process consists of first going arbitrarily far to the right, and then going arbitrarily far to the left, being particularly careful not miss vertices at a specified height $i \in \{0,1\}$ as we travel back to the left. More precisely, we prove the following.
\begin{lem}\label{rough moving}
    Given $N> 0$, there exists a finite word $a = a(N) \in S^*$ such that for any $H \in \mathcal{H}$ and $u \in V(H)$, we have $\delta_H(u,a) = (x,y)$ where $x_v > x \implies d_H(u,v)> N$.
\end{lem}
\begin{lem}\label{fine moving}
    Given $y\in\{0,1\}$ and an integer $k \ge 0$, there exists a finite word $a= a(y,k) \in S^*$ such that for any $H \in \mathcal{H},n \in \Bbb{Z}$, if $u,v$ are connected in $H_{n,n+k}$, $y_v = y$, and $x_u\ge x_v$, then $a$ solves $M_{H,u,v}$. 
\end{lem}
Assuming this, we get our main result as follows. 

\begin{thm} There exists $A \in S^\infty$ such that $A$ solves $\mathcal{H}$ starting at $(0,0)$.
\begin{proof}
    We wish to use Remark~\ref{fstrat}. In particular, we may partition $\mathcal{F} = \bigcup_{H \in \mathcal{H}}\{M_{H,v}:\forall v \in V(H)\}$ into the sets $F_{i,j} = \{M_{H,v}: d_H((0,0),v) = i\,y_v = j\}$ with the integers $i \ge 0$ and $j \in \{0,1\}$.\footnote{It is clear that $d_H$ is non-negative and integer-valued when $u,v$ are connected in $H$. (if they are connected, there is a walk $w_0,\dots$ along $H$ enumerated by the naturals such that $w_0 = u$ and $w_n = v$ for some $n$, in which case $d_H(u,v) \le n$) Meanwhile, by definition of $\mathcal{H}$, $H \in \mathcal{H}$ implies that $H$ is connected and contains $(0,0)$ in its vertex set, thus $v \in V(H) \implies d_H((0,0),v)<\infty$.} We set $X_{i,j} = \{(H,v): M_{H,v} \in F_{i,j}\}$ and remark that this partition is clearly countable.
    
    Now, given $a \in S^*$, we wish to show that there exists $b=b(a,(i,j))$ such that $a\bigstar b$ solves $F_{i,j}$. For each $H$, we set $u_0= u_0(H) = \delta_H((0,0),a)$. If $|a| = n$, then we take $N = n+i$ and note that by triangle-inequality we get \[d_H(u_0,v) =d_H(\delta_H((0,0),a),v)\le d_H(\delta_H((0,0),a),(0,0)) + d_H((0,0),v) \le n +i \]for all $(H,v) \in X_{i,j}$. (this is clear as for each letter in $a$, our distance increases by at most 1) It follows by Lemma~\ref{rough moving} that there exists some $b_1 \in S^*$ such that for all $H$, we get \begin{align*}
        \delta_H(a\bigstar b_1,(0,0)) = \delta_H(u_0,b_1) = u_1\textrm{ where }x_{u_1} &\ge \max_{v\in V(H): d_H(u_0,v)\le N}\{x_v\} \\&\ge  \max_{v\in \in X_{i,j}} \{x_v\}.\tag{1a}\\
    \end{align*}
    
    Next, for each $H$, we set $u_1 = u_1(H) =\delta_H((0,0),a\bigstar b_0)$. Again applying triangle inequality we get that for all $(H,v) \in X_{i,j}$, that $d(u_1,v) \le |a\bigstar b_1| +i =: D$ for some finite $D\ge 0$, as both $a$ and $b_1$ are finite. It is clear that for any $H$, all paths of length $\ell$ starting at $u_1$ are contained in $H_{x_{u_1}-\ell,x_{u_1}+\ell}$. It then follows that for each $(H,v) \in X_{i,j}$ that $u_1$ and $v$ are connected in $H_{x_{u_1}-D,x_{u_1}+D}$. Meanwhile, by $(1a)$, we also have that for $(H,v) \in X_{i,j}$ that $x_{u_1} \ge x_v$. Lastly, we observe that $(H,v) \in X_{i,j} \implies y_v = j$. 
    Hence, by Lemma~\ref{fine moving}, taking $y = j$, $n = x_{u_1}-D\le m = x_{u_1}+D$ we have that there exists a finite $b_2 \in S^*$ $d_H(u_1,v)$ which will solve $M_{H,u_1,v}$ for each $(H,v) \in X'_{i,j}$. It follows that $a \bigstar b_1\bigstar b_2$ solves $F_{i,j}$ as desired, with $b_1\bigstar b_2$ being finite. Hence, by Remark~\ref{fstrat}, we are done.
\end{proof}
\end{thm}

I'm afraid I have not proven my lemmas in a way which is very pleasant to read. In Lemma~\ref*{rough moving}, we prove that we may recursively take $a(1) = R, a(N+1) a(N)^{16N+14M+10}U(LU)^{M+N}R^{2M+2N+1}D (LD)^{M+N}R^{M+N+1}$, where $M= |a(N)|$. In Lemma~\ref*{fine moving}, we prove that we may take $a(1,k) = R^k(UL)^{2k+1}R^{2k+1}L^{4k+1}B_{2k}(k)$, where $b_i = R^iD(LU)^{2k}R^{2k}L^{2k}$ and $B_0(k) = b_0^{2k}, B_{i+1}(k) = (B_i(k)b_{i+1})^{2k}$. Swapping the instances of $U$ and $D$ which appear in $a(1,k)$, we get $a(0,k)$.


\begin{proof}[Proof of Lemma~\ref*{rough moving}]
    We proceed by induction. Our inductive hypothesis is this: for each $N\ge 1$, there exists a finite word $a = a(N) \in S^*$ such that for all $H\in \mathcal{H}$, and any $u \in V(H)$, we have that $x_v > x_{\delta_H(u,a)} \implies d_H(u,v) > N$.
    
    
    For $N = 1$, we have that $a(1) = R$ suffices. Now, given that there exists $a = a(N)$ satisfying our inductive hypothesis for the case of $N$, we wish to show this extends to the case of $N+1$.
    
    We set $a = a(N)$, $M = |a|$. We intend to show that we may take $a(N+1)$ to be $A:= a^{16N+14M+10}U(LU)^{M+N}R^{2M+2N+1}D (LD)^{M+N}R^{M+N+1}$. We make some choice of $H\in \mathcal{H},u \in V(H)$. 
    
    We shall consider the quantities $\Delta_i := x_{\delta_H(u,a^{i+1})}-x_{\delta_H(u,a^i)}$ for $0\le i < 16N+14M+10$. By triangle inequality and the fact that each $s \in S$ can only change the $x$-coordinate by at most $1$, we get that the $x$-coordinate of $\delta_H(u,A)$, $x'$, is bounded like so:
    \begin{align*} x' &\ge x_u + \sum_{i=0}^{16N+14M+9} \Delta_i - |U(LU)^{M+N}R^{2M+2N+1}D (LD)^{M+N}R^{M+N+1}| \\&= x_u+\sum_{i=0}^{16N+14M+9} \Delta_i -7N-7M-4.\\\end{align*}
    
    Case 1: $\sum_{i=0}^{16N+14M+9}\Delta_i \ge 8N+7M+5$
    
    It then follows that $x' \ge x_u+N+1$. We are then immediately done, by the simple upper bound that $d_H(u,v) \ge |x_u-x_v|+|y_u-y_v|$, as then we have $x_v > x'\implies d_H(u,v)\ge 1+x'-x_u \ge N+2> N+1$ as desired.
    
    Case 2: $\sum_{i=0}^{16N+14M+9}\Delta_i \le 8N+7M+4$
    
    We first note that $\Delta_i$ are all non-negative, by our inductive hypothesis.\footnote{We may write $\delta_H(u,a^{i+1}) = \delta_H(\delta_H(u,a^i),a(N))$, thus as the distance from $\delta_H(u,a^i)$ to itself is zero, which is clearly less than $N$, we get that $x$-coordinate of $\delta_H(u,a^{i+1})$ cannot be less than the $x$-coordinate of $\delta_H(u,a^i)$} We moreover note that $\Delta_i=0$ can only occur when $\delta_H(u,a^i)$ does not have an edge which takes us further to the right, as that would imply $d_H(\delta_H(u,a^i), \delta_H(u,a^i)+R) = 1\le N$ and then by definition of $a(N)$ we would get $x_{\delta_H(u,a^{i+1})} \ge x_{\delta_H(u,a^i)+R} = x_{\delta_H(u,a^i)}+1\implies \Delta_i \ge 1$.
    
    It then follows by pigeonhole principal that there must exist some $0\le j<8N+7M+5$ such that $\Delta_{2j}+\Delta_{2j+1} =0$ and thus $(\delta_H(u,a^{2j}),\delta_H(u,a^{2j+1}),\delta_H(u,a^{2j+2}))$ all share the same $x$-coordinate, $x^*$. Applying pigeonhole principal again, since $H$ has at most two vertices with $x^*$ as their $x$-coordinate, we must have that at least two of $(\delta_H(u,a^{2j}),\delta_H(u,a^{2j+1}),\delta_H(u,a^{2j+2}))$ are the same vertex. It then follows that the sequence must be periodic, with period at most 2, meaning $\delta_H(u,a^{2j+2q+r}) = \delta_H(u,a^{2j+r})$ for all $q \ge 0$, $r\in \{0,1\}$.
    
    We next refine this observation. By definition of $\mathcal{H}$, we have that $H$ must extend infinitely far to the right, at least one of $(x^*,0),(x^*,1)$ must have an edge which takes us further to the right. Recalling what we noted two paragraphs ago, both of $\delta_H(u,a^{2j}),\delta_H(u,a^{2j+1})$ cannot have an edge to the right, thus they cannot be distinct as they both have $x^*$ as their $x$-coordinate. Hence, we must have that $\delta_H(u,a^{2j})$ is a fixed point, such that $\delta_H(u,a^{2j}) = \delta_H(\delta_H(u,a^{2j}),a)$. (which of course implies that $\delta_H(u,a^{16N+14M+10}) = \delta_H(u,a^{2j})$ as $2j<16N+14M+10$ and we will be stuck at this fixed point)
    
    We now analyze the scenario further, to deduce that the expected properties of $a(N+1)$ shall hold for $A$ in Case 2, as desired. 
    
    As we have observed above, assuming Case 2 is true, we may choose $0\le j < 16N+14M+10$ such that $\delta_H(u,a^{2j})$ is a fixed point and $\delta_H(u,a^{16N+14M+10}) = \delta_H(u,a^{2j})$. We set $(x^*,y^*) = \delta_H(u,a^{2j})$. Since $(x^*,y^*)$ is fixed under $a(N)$, we must have that $(x^*,y^*)$ does not have an edge taking it further to the right. Since $H$ is connected and extends infinitely far to the right, we must have that $(x^*,1-y^*)$ has edge to its right, and thus $(x^*+1,1-y^*) \in V(H) $. We set $v = (x^*+1,1-y^*)$.
    
    The rest of the argument is essentially the following: 
    \begin{itemize}
        \item We consider the walk $(w_0\dots w_{7N+7M+4})$, with $w_i = \delta_H((x^*,y^*),\alpha_i)$, where $\alpha_i$ is the prefix of $U(LU)^{M+N}R^{2M+2N+1}D (LD)^{M+N}R^{M+N+1}$ having only $i$ letters.
        \item We note that either our $y$-coordinate never changes in our walk, or we reach $v$ at some point in our walk.
        \item In the first case, we deduce that we must also have that $\delta_H(u,A) = \delta_H(u,a^{2j})$ and $d_H(u,v) > N+1$. By the first additional deduction, we get that $x_{\delta_H(u,A)}=x^*$. Meanwhile, it is not hard to see that $x_{v'}\ge x_v =x^*+1\implies d_H(u,v') = d_H(u,v) + d_H(v,v') >N+1$, and so the desired property will hold.  
        \item In the second case, once we first reach $v$ in our walk, we can further observe that our $x$-coordinate will never less than $x_v$ for the rest of the walk. It then follows that $x_{\delta_H(u,a)} \ge x^*+1$. By definition of $a(N)$, and basic induction, we see that $x_{v'} > x^* \implies d_H(\delta_H(u,a^{2j}),v') > N \implies d_H(u,v') > N$, hence $x_{v'} > x^*+1 \implies d_H(u,v') > N+1$, and the desired property again holds. 
    \end{itemize}
    
    Most of this is straightforward geometric intuition. Only difficult claim is the second deduction in the third bullet point. That part breaks into to subcases. If $x^*-x_u\ge N+1$, then we immediately get $d_H(u,v) \ge x^*+1-x_u > N+1$, and the claim follows. Next we consider if $x^*-x_u < N+1$. We outline the argument for $y^*=0$
    
    
    Since $(x^*,y^*)$ does not have an edge to its right, there must exist a unique, minimal path $p$ from itself to $v$; furthermore, the walk associated with $p$ must be encoded by a sequence of actions of the form $L^kUR^{k+1}$ for some $k\ge 0$.\footnote{As $H$ is connected, a minimal walk between the vertices must exist. We then claim that any minimal walk must be of the form $L^kUR^{k+1}$ for some non-negative $k$. Initially we have $y$-coordinate 0 and cannot take an edge downwards or rightwards. If we go left our $y$-coordinate is still 0, since we do not backtrack in a minimal walk we shall not head right, and we still cannot go down, and thus we are in the same situation as before. Eventually we will go up, and reach $y$-coordinate 1, after which point it will only make sense to go right.} (a walk $v_0,\dots ,v_n$ is encoded by a sequence of actions $s_1,\dots s_n \in S^*$ if $v_{i} = \delta_H(v_{i-1},s_i)$ for $1\le i\le n$) 
    
    It is clear from the encoding that $V(p) = \{v\} \cup \{(x,y):k\le x^*-x\ge 0, y\in \{0,1\}\}$. Next, if $k > N\ge x^*-x_u$, then we have $u\in V(p)$, and so there must be a unique minimal path $p'$ connecting $(x^*,y^*)$ and $u$, which is obtained by taking a subgraph of $p$.
    
    If the $y$-coordinate of walk never changes, then we must have $k> N+M$. Assuming $k> N+M$, we shall show that $y_u = y^* = 0$. We suppose the contrary, namely $y_u = 1-y^*$, and shall obtain a contradiction. First, we observe that then $d_H((x^*,y^*),u) > 2M+N+1>|a| \implies \delta_H(u,a) \neq (x^*,y^*)$. At the same time, we will get $d_H((x^*,1-y^*),u) = x^*-x_u \le N$, immediately implying that the $x$-coordinate of $\delta_H(u,a)$ is at least $x^*$. If the $x$-coordinate is exactly $x^*$, then we must have $\delta_H(u,a) = (x^*,1-y^*)$ by our first observation. Since $(x^*,1-y^*)$ has an edge to its right, it would follow that $\Delta_1 \ge 1$, and thus $x_{\delta_H(u,a^2)}> x^*$ at which point it would be impossible to reach the fixed point $(x^*,y^*)$, further iterations of $a(N)$ will never decrease the $x$-coordinate. Similarly, if the coordinate $\delta_H(u,a)$ is greater than $x^*$ then we immediately get that $\delta_H(u,a^{2j})= (x^*,y^*)$ cannot occur, again giving a contradiction. Hence $y_u = y^*$ must hold. 
    
    We then quickly get that $d_H(u,v) = |p|-|p'| \ge 2M+N > N+1$ as desired. Writing $L^kDR^{k+1}$ in place of $L^kUR^{k+1}$, the proof for when $y^* =1$ should also follow.
    
    \hide{
    We now consider 3 subcases.
    
    Subcase 2.1: $x^*-x_u <N+1$, $y^* = 0$
    
    Since $(x^*,y^*)$ does not have an edge to its right, there must exist a unique, minimal path $p$ from itself to $v$; furthermore, the walk associated with $p$ must be encoded by a sequence of actions of the form $L^kUR^{k+1}$ for some $k\ge 0$.\footnote{As $H$ is connected, a minimal walk between the vertices must exist. We then claim that any minimal walk must be of the form $L^kUR^{k+1}$ for some non-negative $k$. Initially we have $y$-coordinate 0 and cannot take an edge downwards or rightwards. If we go left our $y$-coordinate is still 0, since we do not backtrack in a minimal walk we shall not head right, and we still cannot go down, and thus we are in the same situation as before. Eventually we will go up, and reach $y$-coordinate 1, after which point it will only make sense to go right.} (a walk $v_0,\dots ,v_n$ is encoded by a sequence of actions $s_1,\dots s_n \in S^*$ if $v_{i} = \delta_H(v_{i-1},s_i)$ for $1\le i\le n$) 
    
    It is clear from the encoding that $V(p) = \{v\} \cup \{(x,y):k\le x^*-x\ge 0, y\in \{0,1\}\}$. Next, if $k > N\ge x^*-x_u$, then we have $u\in V(p)$, and so there must be a unique minimal path $p'$ connecting $(x^*,y^*)$ and $u$, which is obtained by taking a subgraph of $p$.
    
    Subsubcase 2.1.1: If $k> N+M$, it quickly follows that $y_u = y^* = 0$. We suppose the contrary, namely $y_u = 1-y^*$, and shall obtain a contradiction. First, we observe that then $d_H((x^*,y^*),u) > 2M+N+1>|a| \implies \delta_H(u,a) \neq (x^*,y^*)$. At the same time, we will get $d_H((x^*,1-y^*),u) = x^*-x_u \le N$, immediately implying that the $x$-coordinate of $\delta_H(u,a)$ is at least $x^*$. If the $x$-coordinate is exactly $x^*$, then we must have $\delta_H(u,a) = (x^*,1-y^*)$ by our first observation. Since $(x^*,1-y^*)$ has an edge to its right, it would follow that $\Delta_1 \ge 1$, and thus $x_{\delta_H(u,a^2)}> x^*$ at which point it would be impossible to reach the fixed point $(x^*,y^*)$, further iterations of $a(N)$ will never decrease the $x$-coordinate. Similarly, if the coordinate $\delta_H(u,a)$ is greater than $x^*$ then we immediately get that $\delta_H(u,a^{2j})= (x^*,y^*)$ cannot occur, again giving a contradiction. Hence $y_u = y^*$ must hold.
    
    And so, we get that
    

    Subcase 2.3: $x^*-x_u\ge N+1$
    }
 
\end{proof}

\begin{proof}[Proof of Lemma~\ref*{fine moving}]
    We will demonstrate the case where we take $y = 1$. Swapping the uses of $U$ and $D$, this will in turn hand the case where $y=0$.
    
    We let $T= T(H) = \{x: ((x,0),(x,1)) \in E(H)\}$ be the set $x$-coordinates of vertical edges in $H$. We respectively define the \textit{upper} and \textit{lower} subgraphs of $H$ to denote $H[\Bbb{Z}\times \{1\}]$ and $H[\Bbb{Z}\times \{0\}]$. (the subgraphs induced by only considering the vertices with $y$-coordinate 1 and 0 respectively) We next refer to the disjoint union of the upper and lower subgraphs, as the \textit{horizontal subgraph} of $H$. Whenever we will refer to a connected component, we will me a maximal connected component. So in the natural way, a (upper/lower/horizontal) component of $H$ is maximal connected component in the corresponding subgraph of $H$.
    
    We consider an instance $H,u,n$. We set $m=n+k$. We shall construct a word $a$ in parts, independent of $H,u,n$, and show show that for all $v$ such that $y_v = 1, x_v\le x_u$ where $u,v$ are connected in $H_{n,m}$ that $a$ solves $M_{H,u,v}$.
    
    We set $u_0 = u$. We start with $a_1 = R^k(UL)^{2k+1}, u_1 = \delta_H(u_0,a_1)$. If $y_{u_0}= 1$, then it is clear that $y_{u_1} = 1$, without us even leaving the horizontal component we started in. It thusly follows here that any vertex with $y$-coordinate 1 which is to the left of $u_0$ either belong to the upper component of $u_1$, or is to the left of $u_1$. Meanwhile, if $y_{u_0} = 0$, we have two cases. 
    
    If $y_{u_1} =0$, then we never went upwards, in which case we will have $T\cap [x_{\delta_H(u_0,L^k)},x_{\delta_H(u_0,R^k)}] = \{\}$.\footnote{Since we never went upwards, we have that $\delta_H(u_0,R^k(UL)^{2k+1}) = \delta_H(u_0,R^kL^{2k+1}) $. This also gives us the knowledge that $x_{\delta_H(u_0,R^kL^i)} \not \in T$ for $0\le i < 2k+1$. Hence, noting $\{x_{\delta_H(u_0,R^kL^i)}: 0\le i<2k+1\} \supset \{x_{\delta_H(u_0,R^j)}: 0\le j \le k\}\cup \{x_{\delta_H(u_0,L^i)}: 0< i \le k\}$, the result follows.}\hide{\footnote{Since we never went upwards, we have that $\delta_H(u_0,R^k(UL)^k) = \delta(u_0,R^kL^k)$. This also gives us the knowledge that $x_{\delta_H(u_0,R^kL^i)} \not \in T$ for $0\le i \le k-1$. If $x_{\delta_H(u_0,R^k)}-x_{\delta(u_0,R^kL^k)}<k$ then in the horizontal subgraph of $H$, the connected that $u_0$ belongs to is a path $p$ with less than $k$ edges. It would then follow that the set of $\{\delta(u_0,R^kL^i):0\le i \le k-1\} $ visits each $x$-coordinate of $p$. As each of these $x$-coordinates cannot have a vertical edge, we have that $H=p$ as $H$ is connected. In this case, as $H$ is a horizontal path we clearly have $T=\{\}$. Otherwise, we have $x_{\delta_H(u_0,R^k)}-x_{\delta(u_0,R^kL^k)}=k$ and thus $x_{\delta(u_0,R^kL^i)} = x_{u_1}+k-i$ for $0\le i \le k-1$ and the claim also follows.}} Noting the lower component of $u_0$ in $H_{n,m}$ cannot have any vertices whose $x$-coordinate do not belong to the interval $[x_{\delta_H(u_0,L^k)},x_{\delta_H(u_0,R^k)}]$, we see that in this case $u_0$ is not connected to any vertices with $y$-coordinate 1 in $H_{n.m}$, and thus there is nothing left to consider.
    
    Otherwise, $y_{u_1} = 1$. Hear we claim that any vertex with $y$-coordinate 1 which is to the left of $u_0$ either belong to the upper component of $u_1$, is to the left of $u_1$, or is not connected to $u_0$ in $H_{n,m}$. To see this, consider the $x$-coordinate where we changed $y$-coordinates in our walk, $x'$. If $x' \ge x_{u_0}$, our claim is quite clear, as the coordinates of the upper connected components are non-overlapping intervals. If $x' > $
    
    So in Case 1 and 3, the only cases where we must consider further, we have that $x_{u_0}-2k-1 \le x_{u_1} \le x_{u_0}+k, y_{u_1}$. We take $a_2 = R^{2k+1}L^{4k+1}$ and $u_2 = \delta_H(u_1,a_2)$. We observe that the walk along $H$ starting at $u_1$ and induced by $a_2$ either visits all vertices in the upper component of $u_1$, or at least all those which belong to $H_{n,m}$. Furthermore, we note that either $x_{u_2}\le n\ge x_{u_1}-k$, in which case, we are done, as we visited all vertices in the upper component of $u_1$ which belongs to $H_{n,m}$, and there are no other vertices with $y$-coordinate 1 in $H_{n,m}$ to the left of $u_2$. Otherwise, we must have that $u_2$ does not have an edge to its left and that $x_{u_2} \le x_{u_1}\le x_{u_0}+k\le m+k\le n+2k$. (in fact, more accurately, in this case we can (and should) observe that $\inf\{x\in T: x\ge x_{u_2}\}\le n+2k$)
    
    We next define $b_i = R^iD(LU)^{2k}R^{2k}L^{2k}$ for $0\le i< 2k$. We note that if we start at the left-most vertex of an upper component of $H_{n,n+2k}$, $u'$, that if $\inf\{x\in T: x\ge x_{u'}\} = x_{u'}+ j$, then $i< j \implies \delta_H(u',b_i) = u'$. Meanwhile, considering the walk starting at $u'$ induced by $b_j$, we observe the following. 
    
    Applying $R^jD$ to $u'$, we will just have used the closest vertical edge to the right of $u'$, precisely at $\inf\{x\in T: x\ge x_{u'}\}$, leaving us at $(x_{u'}+j,0) := u''$. If the connected component of $u'$ in $H_{n,n+2k}$ is not equal to the horizontal component of $u'$ in $H_{n,n+2k}$, then $u''$ will still be inside $H_{n,n+2k}$. Applying $(LU)^{2k}$ to $u''$, there are two cases. If $T\cap [x_{u''}-2k,x_{u''}) = \{\}$, then there are no upper components to the left of $u'$ which are connected to $u'$ in $H_{n,n+2k}$. Otherwise, we have that $\sup\{x\in T: x< x_{u'}+j\} = x_{u''}-j'<x_{u'}$ for some $0<j'\le 2k$. ($\sup\{x\in T: x< x_{u'}+j\}<x_{u'}$ follows from the definition of $j$) We then have that applying $(LU)^{j'}$ to $u''$, we will end up at $(x_{u''}-j',1):=u'''$, ending up at the next closest upper component to the left of $u'$. Applying the rest of $(LU)^{j'}$ to $u'''$ in the second case, we will not leave the upper component of $u'''$ and will now be at $u^4$, where $x_{u^4}\ge x_{u'}-2k$. Lastly, if we then apply $R^{2k}$ to $u^4$, we will have explored sufficiently far to the right of $u^4$ in its upper component, and then applying $L^{2k}$, we will have explored sufficiently far to the left in our upper component. (at which point, we will have explored all vertices in the upper component of $u^4$ in $H_{n,n+2k}$, and either end at the left-most vertex of an upper component of $H_{n,n+2k}$, or have gone further left of $H_{n,n_2k}$)
    
    Some looser intuition is, when $i=j$, then the walk induced by $b_i$ goes right until it reaches a vertical edge. Next, then it goes to the left until either its $x$-coordinate drops below $n$, (at which point we no longer care), or until it reaches a vertical, at which point it will go up. It then will sufficiently explore this new upper component, visiting all vertices of the component which also belong to $H_{n,n+2k}$, and then will end at the left-most vertex of the upper component, or no be in $H_{n,n_2k}$, at which point we shouldn't care.
    
    We then define $B_0(k) = b_0^{2k}, B_{i+1}(k) = (B_i(k)b_{i+1})^{2k}$, and shall take $a_3 = B_{2k}(k)$. The intuition behind the strategy is that, supposing we are at a vertex $u'$ that is left-most in its upper component, and $j=\inf\{x\in T: x\ge x_{u'}\}-x_{u'}$. Each time we apply $b_i$, with $i < j$, nothing changes; we call such instances neutral applications. Meanwhile, if we apply $b_j$, we will end up at some new vertex, $u^*$, not missing any vertices with $y$-coordinate 1 that are connected to $u'$ in $H_{n,n+2k}$ along the way. And further more, we will have $x_{u^*}<x_{u'}$. We will call such instances successful applications. The only case we did not analyze is when $i > j$, which we call mysterious applications. This case is not so troublesome, but $B_{2k}(k)$ allows us to totally avoid it, because if we manage to achieve $2k$ success applications before our first mysterious application, we are done, as we have gone $2k$ spaces to the left, and thus fully explored $H_{n,n+2k}$.
    
    It suffices to take $a(1,k) = a_1\bigstar a_2\bigstar a_3$. 
\end{proof}

\newpage

\section{General Case}

Now, let $\mathcal{H}'$ be the set of all connected subgraphs $H$ of $G' = G[ \Bbb{Z}\times \{0,1\}]$ such that $(0,0) \in V(H)$.

In the general case, we graphs where this holds: $\min_{v \in V(H): y_v = 1} \{x_v\} = 1+\min_{v \in V(H): y_v = 0} \{x_v\}$.  Because of this, I believe we must weaken Lemma~\ref{rough moving} as follows.
\begin{lem}\label{rough moving v2}
    Given $N> 0$, there exists a finite word $a = a(N) \in S^*$ such that for any $H \in \mathcal{H}$ and $u \in V(H)$, we have $\delta_H(u,a) = (x,y)$ where $x' > x \implies d_H(u,(x',y'))> N$ or $a$ solves $M_{H,u,(x',y')}$.
\end{lem}I'm quite confident this is true but proving it through induction is an utter mess. See Section~\ref{cs}, for what I believe is the precise inductive hypothesis needed.

We then can prove the general theorem:
\begin{thm} There exists $A \in S^\infty$ such that $A$ solves $\mathcal{H}'$ starting at $(0,0)$.
\begin{proof}
    We wish to use Remark~\ref{fstrat}. In particular, we may partition $\mathcal{F} = \bigcup_{H \in \mathcal{H}'}\{M_{H,v}:\forall v \in V(H)\}$ into the sets $F_{i,j} = \{M_{H,v}: d_H((0,0),v) = i\,y_v = j\}$ with the integers $i \ge 0$ and $j \in \{0,1\}$. We set $X_{i,j} = \{(H,v): M_{H,v} \in F_{i,j}\}$ and remark that this partition is clearly countable.
    
    Now, given $a \in S^*$, we wish to show that there exists $b=b(a,(i,j))$ such that $a\bigstar b$ solves $F_{i,j}$. For each $H$, we set $u_0= u_0(H) = \delta_H((0,0),a)$. If $|a| = n$, then we take $N = n+i$ and note that by triangle-inequality we get $d_H(u_0,v) \le N$ for all $(H,v) \in X_{i,j}$. It follows by Lemma~\ref{rough moving v2} that there exists some $b_1 \in S^*$ such that for all $H$, letting $\Solve_H$ be the vertices $v \in V(H)$ such that $b_1$ solves $M_{H,u_0,v}$, we get   \begin{align*}
        \delta_H(a\bigstar b_1,(0,0)) = \delta_H(u_0,b_1) = u_1\textrm{ where }x_{u_1} &\ge \sup_{v\not \in \Solve_H(u_0,a): d_H(u_0,v)\le N}\{x_v\} \\&\ge  \sup_{v\not \in \Solve_H:(v,H) \in X_{i,j}} \{x_v\}.\tag{1b}\\
    \end{align*}
    We set $X'_{i,j} = \{(H,v)\in X_{i,j}: v \not\in\Solve_H\}$, which corresponds to the mazes $M_{H,v}$ that $a\bigstar b_1$ does not necessarily solve.
    
    Next, for each $H$, we set $u_1 = u_1(H) =\delta_H((0,0),a\bigstar b_0)$. Again applying triangle inequality we get that for all $(H,v) \in X_{i,j}$, that $d(u_1,v) \le |a\bigstar b_1| +i =: D$ for some finite $D\ge 0$, as both $a$ and $b_1$ are finite. It is clear that for any $H$, all paths of length $\ell$ starting at $u_1$ are contained in $H_{x_{u_1}-\ell,x_{u_1}+\ell}$. It then follows that for each $(H,v) \in X_{i,j}$ that $u_1$ and $v$ are connected in $H_{x_{u_1}-D,x_{u_1}+D}$. Meanwhile, by $(1b)$, we also have that for $(H,v) \in X'_{i,j}$ that $x_{u_1} \ge x_v$. Lastly, we observe that $(H,v) \in X_{i,j} \implies y_v = j$. 
    Hence, by Lemma~\ref{fine moving}, taking $y = j$, $n = x_{u_1}-D\le m = x_{u_1}+D$ we have that there exists a finite $b_2 \in S^*$ $d_H(u_1,v)$ which will solve $M_{H,u_1,v}$ for each $(H,v) \in X'_{i,j}$. It follows that $a \bigstar b_1\bigstar b_2$ solves $F_{i,j}$ as desired, with $b_1\bigstar b_2$ being finite. Hence, by Remark~\ref{fstrat}, we are done.
\end{proof}
\end{thm}

\section{Chickenscratch} \label{cs}
\begin{proof}[Beginning of a proof of Lemma~\ref*{rough moving v2}]
    We proceed by induction, and shall in fact prove a slightly stronger statement. Specifically, our inductive hypothesis is this: for each $N\ge 1$, there exists a finite word $a = a(N) \in S^*$ that satisfies the following: 
    For $H\in \mathcal{H}$, and any $u \in V(H)$, let $x''$ be largest $x$-coordinate over all vertices such that $d_H(u,v)\le N$. If there exists $v \in V(H)$ so that $x_v > x''$, then $x_{\delta_H(u,a)} \ge x''$. Otherwise, we must have $x_{\delta_H(u,a)} \ge x''-1$, and for all $(x',y') \in V(H)$, we must have $x' >x_{\delta_H(u,a)}$ implies that $a$ solves $H_{H,u,(x',y')}$. (we note in the second case, that $x_{\delta_H(u,a)}\ge x''-1$ implies that the $x$-coordinate of $\delta_H(u,a)$ is at least the second largest $x$-coordinate achieved in $H$, which we call the second-case-property for short)
    
    
    For $N = 1$, we have that $a(1) = R$ suffices. Now, given that there exists $a = a(N)$ satisfying our inductive hypothesis for the case of $N$, we wish to show this extends to the case of $N+1$. We consider some choice of $H\in \mathcal{H},u \in V(H)$, letting $x''$ be largest $x$-coordinate over all vertices such that $d_H(u,v)\le N+1$.
    
    
    
    Case 1: there exists $w \in V(H)$ such that $x_w > x''$.
    
    We set $a^1 = a(N)$ and recursively define $a^{i+1} = a^i\bigstar a^1$. We then consider $v = \delta_H(u,a^{N+1})$. If $x_v \ge x_u + N+1$, we are done, by the simple upper bound that $d_H(u,(x',y')) \ge |x_u-x'|+|y_u-y'|$ as then we have $x' > x_v \implies d_H(u,(x',y'))\ge 1+|x_v-x_u| \ge N+2$. Otherwise, by pigeonhole principal, we must have that $\Delta_i := x_{\delta_H(u,a^{i+1})}-x_{\delta_H(u,a^i)}<1$ for some $0\le i\le N$.
    
    By the statement of our hypothesis, if $\Delta_i < 0$ for some $i$, then $\delta_H(u,a^{i+1})$ has the second-case-property, meaning $x_{\delta_H(u,a^{i+1})}\ge x_w-1 \ge x''$. It should then follow that the second-case property should always hold going forward, giving our desired inequality. Hence, we are left to consider when we have $\Delta_i = 0$ for some $i$... 
\end{proof}
\section{Random Analogue}

\begin{defn}
    We say a word $a$ solves the $r$-ball of a family of graphs $\mathcal{H}$ starting at $u$, if $a$ solves $\{M_{H,u,v}: H\in \mathcal{H}, v\in V(H) \textrm{ s.t. } d_H(u,v)\le r\}$. 
\end{defn}

\begin{que}
    Under what conditions does a random walk solving $\mathcal{H}$ with probability 1 imply that for all $d$, that with a random walk will solve $\{M_{H,v}: H\in\mathcal{H},d_H(0,v) \le d\}$ with probability 1.
\end{que}
By Remark 3.3 of the other constructive paper:
\begin{rmk}
    For each $H$, we can associate an indicator function $f_H:e\mapsto \chi_{E(H)}(e)$. We then get a compact metric space by equipping the set of functions $f_H$ with the product topology. For a family of graphs $\mathcal{H}$ such that for all paths, $p$, the set $S_p:=\{f_H:p\le H \in \mathcal{H}\}$ is compact, then if there exists $a \in S^\infty$ which solves $\mathcal{H}$, then for every $r$ there exists a finite prefix $a'$ of $a$ which solves the $r$-ball of $\mathcal{H}$. (we will call this property of a family path-compactness)
\end{rmk}

\begin{rmk} If there exists a countable partition $\mathcal{F} = K \cup \bigcap_{i=1}^\infty F_i$ such that for each $F_i$ and $a \in S^*$ there exists $b= b(a,i) \in S^*$ such that $a\bigstar b$ solves $F_i$, and $K$ is a path compact family of graphs (in the sense of the previous remark) which can be solved by $a \in S^\infty$, then there exists $a' \in S^\infty$ so that $a'$ solves $\mathcal{F}$.
\end{rmk}

\section{Further Questions}

\begin{defn}
    For a set $T \subset \Bbb{Z}$, we denote $\ell(T)$ as the set of all labelings, $f:\Bbb{Z}\to T$. 
\end{defn}

Given a labeling $f: \Bbb{Z}\to \Bbb{Z}$, we may define a walk, $w_0,\dots $, along $\Bbb{Z}$ for words $a \in \{L,R,M\}^\infty$, such that $w_0 =0$, and \[\begin{cases}w_{i+1} = w_i -1& \text{if } a_i = L\\ w_{i+1} = w_i +1& \text{if } a_i = R\\w_{i+1} = w_i +f(w_i)& \text{if } a_i = M.\\\end{cases}\]We say a word $a \in \{L,R,M\}^\infty$ solves a labelling $f$ if for each $n \in \Bbb{Z}$, there exists some $t$ such that $w_t = n$ and $a_t = n$. One may think of letter $M$ being the action of marking a vertex. We would like to mark all the vertices of $\Bbb{Z}$, but doing so affects our location, depending upon the labelling $f$.

\begin{que}
    For each finite $T$, does there exist a word $a \in \{L,R,M\}^\infty$ such that $a$ solves each $f \in \ell(T)$?
\end{que}

The motivation for this is as follows. Consider a graph $H$ formed by infinite horizontal line $\Bbb{Z}\times \{0\}$ and for each $n\in \Bbb{Z}$ we glued a finite graph $H_n$ to $(n,0)$. Lets assume that the graphs $H_n$ do not get arbitrarily large, and were disjoint. The family of graphs $H$ created in this way would be a nice case to handle. If the above holds true, we should (probably) get this as follows. 

For a specific $H$, if the graphs $H_n$ do not get arbitrarily large, there are finitely many such graphs in the set $G_H :=\{H_n:n \in \Bbb{Z}\}$. There should then exist a universal finite word, $b \in S^*$, which, starting at $(n,0)$ should fully explore any $G \in G_H$ glued to $(n,0)$ and ends us up at another point at the line, say $(n+i,0)$.* Here, because $b$ is finite, $i$ should be a function of structure of $H$ locally around $(n,0)$, meaning there are finite many offsets $i$ that can occur. This corresponds to a labeling $f:n\mapsto i$ with finite image. We can then use the universal word which solves all $f \in \ell(T)$ to fully explore the line and all the graphs glued to it, which would work for all $H'$ where $G_{H'} \subseteq G_H$. Combining this with Remark~\ref{fstrat}, and increasing sequence of finite graphs $G_i$ whose limit is the set of all finite graphs, we get our result. 

*Here is the only part I am not 100 percent certain of. While there definitely exists a universal word that explores any glue graph $G\i G_H$ and then exits, it may take a little work to make sure that we end at the line, and don't get into some other neighboring graph and stick inside  there.

Also, the current partial results we have come up with (including some which are not document) seem to boil down to a clever application of solving all $f\in \ell(\{0,1\})$ and then applying Remark~\ref{fstrat}, so this problem seems quite relevant in this regard to. 

We note without proof that if there is a word solving $\ell(T)$, then there is a word solving $\ell(T+k)$ and $\ell(Tk)$ for $k \neq 0$.(using typical sumset/product set definitions. Here is the smallest problem we don't know how to prove:

\begin{que}
    Does there exist a word $a \in \{L,R,M\}^\infty$ such that $a$ solves each $f \in \ell(\{-1,0,1\})$?
\end{que}

We believe it may be sufficient to consider the following variant of random walk. We choose $s_i$ uniformly from random from $\{L,R\}$, and $X_i$ uniformly at random from $\{1, \dots , g(i)\}$ for some function $g:\Bbb{N} \to \Bbb{N}$ such that $g(n)\to \infty$ as $n\to \infty$ and $\sum_{n=1}^\infty \frac{1}{g(n)} = \infty$. Then, with the correct choice of $g$, letting $a_1 = s_1^{X_1}, a_{i+1} = a_i\bigstar M\bigstar s_i^{X_i}$, and $a$ be the direct limit of this sequence, that almost surely $a$ solves all $f \in \ell(\{-1,0,1\})$. Perhaps we may even get that this construction for random words almost surely solves all $f \in \ell(T)$ for finite $T$.

We also are curious about using such words to potentially produce words which solve all subgraphs of $\Bbb{Z}^2$. (here of course we remove all the instances of $M$ and sample $s_i$ uniformly from $\{L,R,U,D\}$)

Even easier problem I'm still not quite sure how to prove:

\begin{que}
    Does there exist a word $a \in \{L,R,M\}^\infty$ such that $a$ solves each $f \in \ell(\{-1,0,1\})$ such that $\supp(f)\subseteq \{2^n:n \in \Bbb{N}\}$?
\end{que}

\end{document}